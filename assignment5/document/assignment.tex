\documentclass[12pt,a4paper]{article}
\usepackage[utf8]{inputenc}
\usepackage{amsmath}
\usepackage{amsfonts}
\usepackage{amssymb}
                          
\begin{document}
\begin{center}
  \Large Assignment 5 \\
  \large Distributed Algorithms
\end{center}
\begin{flushright}
  \begin{tabular}{ll}
    \textbf{Group 11} \\
    Dan Drewes        \\ 
    Manuela Hopp      \\ 
    John Mercouris    \\
    Malte Siemers     \\
  \end{tabular} 
\end{flushright}



\section*{Exercise 5.1: Byzantine Generals}
\begin{enumerate}
% Describe the agreement of the byzantine generals and explain the recursive algorithm for oral messages.
\item[a)] The aggreement of the byzantine generals discribes a set of problems of message based communication. For a set of communicating entities, i.e. processes in a distributed system, some difficulties arise when they try to find a consensus or a majority of a vote if faulty processes, message loss or delay are taken into account. The algorithm discribed in the  lecture solves the problem of faulty nodes. For a set of n = 3m+1 processes, m being the amount of tolerated faulty processes, it can be shown that the non faulty processes can find a common decision. A commanding process sends a message, 0 or 1, to every other process and  they then relay the message to every other process that didnt receive that specific message yet, excluding the commander. For three processes and one commander, one of which is faulty, the relay has to happen only once for a consensus, for every note then received three instances of the message of which one is faulty. For n = 3m+1 processes messages must be relayed m times in a recursive way, meaning the message invokes an other instance of it self upon reception if it wasnt relayed m times yet. Every process then has a tree of messages that routed through the network one way or the other and decides for the command occuring mostly in its tree.
\item[b)]
  \begin{enumerate}
  % n = 4 (4 generals), m = 1 (one randomly chosen general is faulty)
  \item[(a)] -
  % n = 7 (7 generals), m = 2 (two randomly chosen generals are faulty).
  \item[(b)] -
  % n = 10 (10 generals), m = 3 (three randomly chosen generals are faulty).
  \item[(c)] -
  \end{enumerate}
\end{enumerate}

\section*{Exercise 5.2: Self-Stabilizing Spanning Tree}
\begin{enumerate}

\item[a)] What is the behavior of the algorithm in case of the following faults:
  \begin{itemize}
  \item Failure or reintegration of nodes or links: When a node or link fails, depending on where in the tree it fails can cause different behavior. Imagine that we have two semi-independent graphs that are linked by a single node. If this node fails, we now have two distinct graphs which will elect their own leaders when they hit their timeouts. This will lead to two independent systems. When they reconnect, there will be a conflict of leader, and they will reperform an election algorithm. In this way, the algorithm will always heal itself to have a logical leader, but depending on how the network breaks, it may lead to unexpected/unpreventable behavior.
  \item Corruption of local data structures: If the local data structures storing information are corrupt, it is potentially worst than the node being knocked out altogether. It may send information about who the leader is incorrectly to children nodes, it may cause a perpetual `rehelealing` loop of the program to occur.
  \item Corruption or loss of messages: When messages are corrupt or lost, the timeout in nodes will be triggered forcing a re-election. If messages eventually can make it through, then the network will heal itself into its' original state. The only downside will be that messages were sent with the intent of a reorganization of the span structure when it was not necessary. If it is shown that the timeout is persistent, or a consistent failure, the network will again heal itself by forming a new spanning tree to account for the unreachable set of nodes.
  \end{itemize}
% In large dynamic systems, changes of the topologies are not an unusual behavior, but rather normal. What is the impact on the proposed algorithm?
\item[b)] The tree will potentially need to rearrange itself if the topology severs critical node connections that constitute the spanning tree. If, on the other hand the topology changes and we remove all connections that are not relevant to the spanning tree, AKA all links that are not spanning tree links, then we will be able to maintain the spanning tree without reorganization. In the event of a topology reorganization that breaks the spanning tree, the algorithm will detect this after the timeout and reissue an election which will restructure the tree as well.
% Implement the selfstabilizing spanning tree algorithm! Use setTimeout(..) and timeout(..) in order to realize heartbeats.
\item[c)]
  \begin{enumerate}
  % Choose an appropriate value for the timeout and explain why.
  \item[(a)] The timeout is based on the number of levels to the root node and the maximally tolerated delay for a message to send. The delay is configurable in the teachnet configuration. For example, if our maximum delay that we'll tolerate between two nodes is 30ms, and we find that we are on the 3rd level of the tree, and we know that heartbeats occur every 10ms, we will have a timeout of 100ms. This will allow for the heartbeat to be initiated, and all delays between all nodes to the sending node to be at their maximum, and then to finally recieve a message before the next one is triggered, assuming worst case after worst case scenario.
  % Simulate node failures after a configurable amount of time (at least one node failure). Timeouts could be used for this as well: In this case, after the timeout appeared, the node will be silent from now on, i.e., it swallows all incoming messages and stays calm (no heartbeats).
  \item[(b)] -
  % Simulate that the faulty nodes recover from their faulty state and reintegrate themselves in the network. Simulate at least one node reintegration.
  \item[(c)] -
  \end{enumerate}
\end{enumerate}


\end{document}
