\documentclass[12pt,a4paper]{article}
\usepackage[utf8]{inputenc}
\usepackage{amsmath}
\usepackage{amsfonts}
\usepackage{amssymb}
                          
\begin{document}
\begin{center}
  \Large Assignment 5 \\
  \large Distributed Algorithms
\end{center}
\begin{flushright}
  \begin{tabular}{ll}
    \textbf{Group 11} \\
    Dan Drewes        \\ 
    Manuela Hopp      \\ 
    John Mercouris    \\
    Malte Siemers     \\
  \end{tabular} 
\end{flushright}



\section*{Exercise 5.1: Byzantine Generals}
\begin{enumerate}
% Describe the agreement of the byzantine generals and explain the recursive algorithm for oral messages.
\item[a)] Placeholder Text
\item[b)]
  \begin{enumerate}
  % n = 4 (4 generals), m = 1 (one randomly chosen general is faulty)
  \item[(a)] Text
  % n = 7 (7 generals), m = 2 (two randomly chosen generals are faulty).
  \item[(b)] Text
  % n = 10 (10 generals), m = 3 (three randomly chosen generals are faulty).
  \item[(c)] Text
  \end{enumerate}
\end{enumerate}

\section*{Exercise 5.2: Self-Stabilizing Spanning Tree}
\begin{enumerate}

\item[a)] What is the behavior of the algorithm in case of the following faults:
  \begin{itemize}
  \item Failure or reintegration of nodes or links:
  \item Corruption of local data structures:
  \item Corruption or loss of messages:
  \end{itemize}
% In large dynamic systems, changes of the topologies are not an unusual behavior, but rather normal. What is the impact on the proposed algorithm?
\item[b)] Text
% Implement the selfstabilizing spanning tree algorithm! Use setTimeout(..) and timeout(..) in order to realize heartbeats.
\item[c)]
  \begin{enumerate}
  % Choose an appropriate value for the timeout and explain why.
  \item[(a)] Text
  % Simulate node failures after a configurable amount of time (at least one node failure). Timeouts could be used for this as well: In this case, after the timeout appeared, the node will be silent from now on, i.e., it swallows all incoming messages and stays calm (no heartbeats).
  \item[(b)] Text
  % Simulate that the faulty nodes recover from their faulty state and reintegrate themselves in the network. Simulate at least one node reintegration.
  \item[(c)] Text
  \end{enumerate}
\end{enumerate}


\end{document}
