\documentclass[12pt,a4paper]{article}
\usepackage[utf8]{inputenc}
\usepackage{amsmath}
\usepackage{amsfonts}
\usepackage{amssymb}
\begin{document}
\begin{center}
  \Large Assignment 1\\
  \large Distributed Algorithms
\end{center}
\begin{flushright}
  \begin{tabular}{ll}
    \textbf{Group 11} \\
    Dan Drewes     	  \\ 
    Manuela Hopp       \\ 
    John Mercouris    \\
    Malte Siemers     \\
  \end{tabular} 
\end{flushright}

\subsection*{Exercise 1.1: Warming Up}
\paragraph*{a)} % What is the difference between a distributed system and a parallel computer?
A distributed system is a system of independent machines that work together and communicate with each other by exchanging messages. Parallel computing is more along the lines of a single 'machine' executing a task. Additionally, the processors of a parallel computer can access shared memory.
\paragraph{b)} % Why do we use distributed systems although they are complicated?
We use distributed systems because they are cheap, can be very quick, are reliable, and can grow in a linear fashion. This makes it very attractive for several economic reasons. 
\paragraph{c)} % What are the differences between the synchronous model, the asynchronous model and the atom model?
In a synchronous model, the length of actions and the maximum delay of messages is restricted and known. In an asynchronous model, actions and messages can take arbitrarily long or the limits are unknown. The atom model is a partially synchronized model where the delay of messages is unknown and the length of an action is assumed to be timeless.

\subsection*{Exercise 1.2: Topologies}
% Consider a hypercube with dimension d
\paragraph{a)} % Given two arbitrary nodes u and v from the hypercube. How many shortest paths are there between u and v?
If the distance between $u$ and $v$ is $\delta$, then there are $\delta!$ shortest paths between $u$ and $v$ as we can correct the bits that differ between $u$ and $v$ in an arbitrary order.
\paragraph{b)} % How many different node pairs are there, that are connected with shortest path of length k?
For each of the $2^d$ nodes there are $\binom{d}{k}$ nodes with distance $k$.
Thus, there are $\frac{1}{2}\cdot2^d\cdot\binom{d}{k}=2^{d-1}\binom{d}{k}$ pairs of nodes that are connected with shortest path of length $k$.
\paragraph{c)} % Given the broadcast algorithm from the lecture to produce spanning trees on the hypercube. How many different spanning trees could be generated from the same start node by varying the dimension used for sending?
Each dimension must be used exactly once but the order is arbitrary. Therefore, on a $d$-dimensional hypercube, $d!$ different spanning trees could be created from one start node.
\paragraph{d)} % Would it be possible to do multiple broadcasts in parallel with the previous algorithm from the same start node? i.e. in the unit time model there exist at most one message on an edge 
Yes, it would be possible.

\subsection*{Exercise 1.3: Distribution of Information}
\paragraph{a)} Flooding in a bidirectional ring sends $n+2$ messages more than broadcast in a unidirectional ring. Since the number of nodes and edges in a ring are equal, we have $4e-2n+2=2n+2$ messages compared to $n$ in the broadcast case.
\paragraph{c)}

\subparagraph{a)}
\begingroup
\leftskip1em
In a bidirectional ring there is no message reduction through the improvement with taboo sets. Every set of taboo nodes that is sent, only contains nodes that are behind the activator from the perspective of the reciever. So the recieving node wouldn`t have sent any Explorers into this direction anyway.
\par
\endgroup
\subparagraph{b)}

\begingroup
\leftskip1em
In a binary X-tree with heigth $h$ the improved Echo algorithm tells every child node to send no message to its sibling (the one node with the same parent node). Therefore the amount of reduced messages is equal to the number of edges between siblings in such a tree, which is half the number of nodes minus one: $\dfrac{2^{h+1}-1-1}{2}=2^{h}-1$. If $h=0$ there are obviously no messages send.
\par
\endgroup

\subsection*{Exercise 1.4: Election}
\paragraph{b)} % There is a precondition for the algorithm of Chang and Roberts that all node IDs have to be unique (no duplicate IDs). Assume, we drop this precondition and allow multiple nodes to have the same ID.
\subparagraph{a)} % Does the algorithm still work properly? Please provide a reasonable answer.
\subparagraph{b)} % In which cases does the algorithm still deliver a proper result? Explain at least two cases.
\end{document}