\documentclass[12pt,a4paper]{article}
\usepackage[utf8]{inputenc}
\usepackage{amsmath}
\usepackage{amsfonts}
\usepackage{amssymb}
\begin{document}
\begin{center}
  \Large Assignment  \\
  \large Distributed Algorithms
\end{center}
\begin{flushright}
  \begin{tabular}{ll}
    \textbf{Group 11} \\
    Dan Drewes        \\ 
    Manuela Hopp      \\ 
    John Mercouris    \\
    Malte Siemers     \\
  \end{tabular} 
\end{flushright}


\section*{Exercise 4.1: Physical Clocks}
  \subsection*{a)} % Synchronization
  	\paragraph{a)}
  	\paragraph{b)}
  	\paragraph{c)}
  \subsection*{b)} % Clock condition
    \paragraph{a)} The local clocks my never be turned back and may never halt because otherwise an event A that happened after an event B could have an earlier or the same timestamp and therefore the order of events would not be preserved.
  	\paragraph{b)} Each pair of clocks must either show the same time or the difference must be known such that the order of events that happened on different machines can be determined.
  	\paragraph{c)} It cannot be determined from the timestamps alone whether two events causally depend on each other. Therefore, the utilization of physical clocks does not fulfill the converse of the clock condition.
  	
\section{Exercise 4.2: Vector Clocks}
	\subsection*{a)} % Causal Order
	\subsection*{b)} % Order Relation
		\paragraph{a)}
		\paragraph{b)}
		\paragraph{c)}

\end{document}
